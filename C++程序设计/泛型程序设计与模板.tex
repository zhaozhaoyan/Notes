\documentclass{article}  
\usepackage{CJKutf8}
\usepackage{minted}
\begin{document} 
\hfuzz=\maxdimen
\tolerance=10000
\hbadness=10000
\begin{CJK}{UTF8}{gbsn}  
\title{第九章  泛型程序设计与模板}
\author{}
\date{}
\maketitle
\part*{一、函数模板}
\subsection*{(1)函数模板的原理}
\begin{minted}{c++}
    int max(int a,int b){return (a>b)?a:b;}
    float max(float a,float b){return (a>b)?a:b;}
    char max(char a,char b){return (a>b)?a:b;}
\end{minted}
\subparagraph*{}
以上程序处理类型一模一样,引入函数模板
\subparagraph*{}
1.类型作为参数,有三个关键字:template, typename,class
\subparagraph*{}
2.定义:函数模板用于生成函数
\begin{minted}{c++}
    template<class 类型参数1,class类型参数2,...>
    返回值类型  模板名(形参表)
    {
        函数体
    }
    //class关键字也可以用typename代替
\end{minted}
\begin{minted}{c++}
    //swap模板
    template<class T>
    void swap(T & x,T & y)
    {
       T tmp=x;
       x=y;
       y=tmp;
    }
\end{minted} 
\begin{minted}{c++}
    //函数模板(比较大小)
    template<class T>
    T max(T a,T b)
    {
       return(a>b)?a:b;
    }
\end{minted}
\begin{minted}{c++}
    //模板函数
    int ival=max(100,99);
    char ival=max<char>('A','B');
\end{minted}
\begin{minted}{c++}
    //swap模板(用typename来定义)
    template<typename T>
    void swap(T & x,T & y)
    {
       T tmp=x;
       x=y;
       y=tmp;
    }
\end{minted}
\begin{minted}{c++}
    //模板函数
    int x=20,y=30;
    swap<int>(x,y);
\end{minted}
\subparagraph*{}
3.变量作为模板参数
\begin{minted}{c++}
    template<int size>
    void display()
    {
       cout<<size<<endl;
    }
\end{minted}
\begin{minted}{c++}
    display<10>();
\end{minted}
\subparagraph*{}
4.多参数函数模板
\begin{minted}{c++}
    template<typename T,typename C>
    void display(T a,C b)
    {
       cout<<a<<""<<b<<endl;
    }
\end{minted}
\begin{minted}{c++}
    int a=1024;
    string str="hello world!";
    disply<int,string>(a,str);
\end{minted}
\subparagraph*{}
5.typename和class可以混用
\begin{minted}{c++}
    template<typename T,class U>
    T minus(T *a,U b);
\end{minted}
\begin{minted}{c++}
    template<typename T,int size)
    void display(T a)
    {
      for(int i=0;i<size;i++)
      cout<<a<<endl;
    }
    dispaly<int,5>(15);
\end{minted}
\subsection*{(2)函数模板与重载}
\begin{minted}{c++}
    template<typename T>
    void display(T a);//只有一个参数
    template<typename T>
    void display(T a,T b);//有两个参数,个数不同
    template<typename T,int size>
    void display(T a);//一个参数,一个变量
\end{minted}
\subparagraph*{}
实例化为:
\begin{minted}{c++}
    dispaly<int>(10);
    dispaly<int>(10,20);
    dispaly<int,5>(30);//在定义出函数模板的时候,需要注意,函数模板本身并不是相互重载的关系,因为在内存中仅仅将函数模板定义出来,而不会出现代码,只有使用的时候,才会产生出相应的模板
\end{minted}
\subsection*{(3)函数模板的编码实现}
\begin{minted}{c++}
#include<iostream>
#include<stdlib.h>
using namespace std;
//函数模板,要求定义函数模板display
template<typename T>
void display(T a)
{
  cout<<a<<endl;
}
template<typename T,class S>
void display(T t,S s)
{
  cout<<t<<endl;
  cout<<s<<endl;
}
template<typename T,int KSize>
void display(T a)
{
  for(int i=0;i<KSize;i++)
  {
    cout<<a<<endl;
  }
}
int main()
{
  //display<double>(10.89);
  //display<int,double>(5,8.3);
  display<int,5>(6);//输出五个6
  return 0;
}
\end{minted}
\part*{二、类模板}
\subsection*{(1)类模板的原理}
\begin{minted}{c++}
    template<class T>
    class MyArray
    {
       public:
          void display()
          {......}
       private:
          T *m_pArr;
    };
\end{minted} 
\subparagraph*{}
类模板的成员函数放到类模板定义外面写时:
\begin{minted}{c++}
    template<class T>
    void MyArray<T>::display()
    {
       ......
    };
\end{minted} 
\begin{minted}{c++}
    int main()
    {
       MyArray<int>arr;
       arr.display();
       return 0;
    }
\end{minted} 
\subparagraph*{}
类模板中有多个参数时
\begin{minted}{c++}
    template<typename T>
    class Container
    {
       public:
          void display;
       private:
          T m_obj;
    };
\end{minted} 
\subparagraph*{}
类外定义时:
\begin{minted}{c++}
    template<typename T,int KSize>
    void Container<T,KSize>::display()
    {
       for(int i=0;i<KSize;i++)
       {
           cout<<m_obj<<endl;
       }
    }
    int main()
    {
      Container<int,10>ct1;
      ct1.display();
      return 0;
    }
\end{minted} 
\subparagraph*{}
特别提醒:因为IDE环境问题,以及其他问题,模板代码不能分离编译
\subsection*{(2)类模板的编码实现}
\begin{minted}{c++}
#include<iostream>
#include<string>
using namespace std;
//类模板,定义类模板MyArray,成员函数:构造函数,析构函数,display函数
                      //数据成员:m_pArr
template<typename T,int KSize,int KVal>
class MyArray
{
public:
  MyArray();
  ~MyArray()
  {
    delete[]m_pArr;
    m_pArr=NULL;
  }
  void display();
private:
  T *m_pArr;//模板参数T
};
template<typename T,int KSize,int KVal>
MyArray<T,KSize,KVal>::MyArray()
{
  m_pArr=new T[KSize];
  for(int i=0;i<KSize;i++)
  {
    m_pArr[i]=KVal;
  }
}
template<typename T,int KSize,int KVal>
void MyArray<T,KSize,KVal>::display()
{
  for(int i=0;i<KSize;i++)
  {
    cout<<m_pArr[i]<<endl;
  }
}
int main()
{
  MyArray<int,5,6>arr;//一个int型数组,数组长度为5,数组元素都是6,有5个6的数组
  arr.display();
  return 0;
}
\end{minted}
\subparagraph*{}
练习题:定义一个矩形类模板,该模板中含有计算矩形面积和周长的成员函数,数据成员为矩形的长和宽。
\begin{minted}{c++}
#include <iostream>
using namespace std;

/**
 * 定义一个矩形类模板Rect
 * 成员函数:calcArea()、calePerimeter()
 * 数据成员:m_length、m_height
 */
template <typename T>
class Rect
{
public:
   Rect(T length, T height);
   T calcArea();//面积
   T calePerimeter();//周长
public:
	T m_length;//长
	T m_height;//宽
};

/**
 * 类属性赋值
 */
template <typename T>
Rect<T>::Rect(T length, T height)
{
	m_length = length;
	m_height = height;
}

/**
 * 面积方法实现
 */
template <typename T>
T Rect<T>::calcArea()
{
	return m_length * m_height;
}

/**
 * 周长方法实现
 */
template <typename T>
T Rect<T>::calePerimeter()
{
	return ( m_length + m_height) * 2;
}

int main(void)
{
	Rect<int> rect(3, 6);
	cout << rect.calcArea() << endl;
	cout << rect.calePerimeter() << endl;
	return 0;
}
\end{minted}
\end{CJK}
\end{document}