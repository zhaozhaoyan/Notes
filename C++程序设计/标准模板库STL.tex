\documentclass{article}  
\usepackage{CJKutf8}
\usepackage{minted}
\begin{document} 
\hfuzz=\maxdimen
\tolerance=10000
\hbadness=10000
\begin{CJK}{UTF8}{gbsn}  
\title{第十章  标准模板库STL}
\author{}
\date{}
\maketitle
\part*{一、标准模板库STL}
\section*{1.STL中的基本概念}
\subparagraph*{}
标准模板库(Standard Template Library)就是一些常用的数据结构和算法的模板的集合。有了STL,不必再写大多的标准数据结构和算法,并且获得非常高的性能。
\subparagraph*{}
1.容器:可容纳各种数据类型的通用数据结构,是类模板;
\subparagraph*{}
2.迭代器:可用于依次存取容器中的元素,类似于指针;
\subparagraph*{}
3.算法:用来操作容器中的元素的函数模板
\subparagraph*{}
1)sort()来对一个vector中的数据进行排序
\subparagraph*{}
2)find()来搜索一个list中的对象
\subparagraph*{}
算法本身与他们的操作的数据的类型无关,因此他们可以在从简单数组到高度复杂的任何数据结构上使用。
\begin{minted}{c++}
    int array[100];
    sort(array,array+70);//将前70个元素排序
\end{minted}
\subparagraph*{}
该数组就是容器,而int *类型的指针就可以作为迭代器,sort算法作用于数组,对其进行排序。
\section*{2.容器概述}
\subparagraph*{}
可以用于存放各种类型的数据(基本类型的变量,对象等)的数据结构,都是类模板,分为三种:
\subparagraph*{}
1)顺序容器:vector,deque,list;
\subparagraph*{}
2)关联容器:set,mutiset,map,multimap;
\subparagraph*{}
3)容器适配器:stack,queue,priority\_queue;
\subsection*{(1)vector向量}
\subparagraph*{}
头文件<vector>
\subparagraph*{}
1.本质:对数组的封装。
\subparagraph*{}
2.特点:能够在随机读取数据的时候在常数时间内完成
\subparagraph*{}
3.初始化vector对象的方式
\begin{minted}{c++}
    vector<T> v1;      //vector保存类型为T的对象。默认构造函数v1为空
    vector<T> v2(v1);  //v2是v1的一个副本
    vector<T> v3(n,i); //v3包含n个值为i的元素
    vector<T> v4(n);   //v4包含有值初始化元素的n个副本
\end{minted}
\subparagraph*{}
具体代码:
\begin{minted}{c++}
    vector<int>ivec1;
    vector<int>ivec2(ivec1);
    vector<string>svec1;
    vector<string>svec2(ivec);
    vector<int>ivec4(10,-1);     //10个-1的元素初始化ivec4
    vector<string>svec(10,"hi"); //10个hi字符串初始化svec
\end{minted}
\subparagraph*{}
4.vector常用函数
\begin{minted}{c++}
    empty()         //判断向量是否为空
    begin()         //返回想向量迭代器首元素
    end()           //返回向量迭代器末元素的下一个元素
    clear()         //清空向量
    front()         //第一个数据
    back()          //最后一个数据
    size()          //获得向量数据大小 
    push_back(elem)//将数据插入向量尾
    pop_back()     //删除向量尾部数据 
\end{minted}
\subparagraph*{}
实例如下:
\begin{minted}{c++}
   int main(void)
   {
      vector<int>vec;
      vec.push_back(10);      //当前元素尾部插入一个元素10
      vec>push_pop();         //抹掉
      cout<<vec.size()<<endl; //输出数据大小
      return 0;
    }
\end{minted}
\subparagraph*{}
遍历数组:
\begin{minted}{c++}
   for(int k=0;k<vec.size();k++)
   {
     cout<<vec[k]<<endl;
   }                        //也可以用迭代器来遍历数组
\end{minted}
\subsection*{(2)迭代器iterator}
\subparagraph*{}
1)用于指向顺序容器和关联容器中的元素
\subparagraph*{}
2)迭代器的用法和指针类似
\subparagraph*{}
3)有const和非const两种
\subparagraph*{}
4)通过迭代器可以读取它指向的元素
\subparagraph*{}
5)通过非const迭代器还能修改其指向的元素
\subparagraph*{}
6)定义方法:
\begin{minted}{c++}
   容器类名::iterator  变量名:
   容器类名::const_iterator  变量名
   * 迭代器变量名   //访问一个迭代器指向的元素
\end{minted}
\subparagraph*{}
迭代器遍历数组元素:
\begin{minted}{c++}
   int main(void)
   {
      vector vec;
      vec.push_back("hello");
      vector<string>::iterator citer=vec.begin();
      for(;citer!=vec.end();citer++)
         {cout<<*citer<<endl;}
      return 0;
   }    
\end{minted}
\subsection*{(3)链表list}
\subparagraph*{}
头文件<list>
\subparagraph*{}
1.特点:数据插入的速度快,双向链表,元素在内存不连续存放,在任何位置增删元素都能在常数时间内完成。不支持随机存取。
\subparagraph*{}
2.使用方法:与vector使用方法相似
\subsection*{(4)关联容器}
\subparagraph*{}
1.元素是排序的
\subparagraph*{}
2.插入任何元素,都按照相应的规则来确定其位置
\subparagraph*{}
3.在查找时具有非常好的性能
\subparagraph*{}
4.通常以平衡二叉树方式实现,插入和检索的时间都是0(log(N))
\subsection*{(5)映射map}
\subparagraph*{}
头文件<map>
\subparagraph*{}
通过map的键key找到它的值value
\subparagraph*{}
程序示例1:
\begin{minted}{c++}
   map<int,string>m;
   pair<int,string>p1(10,"shanghai");//键为10,值为shanghai
   pair<int,string>p2(20,"beijing");
   m.insert(p1);
   m.insert(p2);
   cout<<m[10]<<endl;
   cout<<m[20]<<endl;
\end{minted}
\subparagraph*{}
程序示例2:
\begin{minted}{c++}
   map<string,string>m;
   pair<string,string>p1("S","shanghai");//键为"S",值为shanghai
   pair<string,string>p2("B","beijing");
   m.insert(p1);//将p1插入m里面
   m.insert(p2);
   cout<<m["S"]<<endl;
   cout<<m["B"]<<endl;
\end{minted}
\subparagraph*{}
代码实例一(vector用法):
\begin{minted}{c++}
#include<iostream>
#include<stdlib.h>
#include<vector>
#include<list>
#include<map>
//通过使用标准模板库,学习vector用法
using namespace std;
int main()
{
  vector<int>vec;
  vec.push_back(3);//从当前向量的尾部插入的
  vec.push_back(4);
  vec.push_back(6);
  //vec.pop_back();//将尾部的元素删除了
  //cout<<vec.size()<<endl;
/*for(int i=0;i<vec.size();i++)//遍历数组
{
  cout<<vec[i]<<endl;
}*/
/*vector<int>::iterator itor=vec.begin();
//cout<<*itor<<endl;//通过指针打印
for(;itor!=vec.end();itor++)//初始条件就是 =vec.begin()//最后一个元素的下一个位置
{
  cout<<*itor<<endl;
}*/
cout<<vec.front()<<endl;//第一个数据
cout<<vec.back()<<endl;//最后一个数据
  return 0;
}  
\end{minted}
\subparagraph*{}
代码实例二(list用法):
\begin{minted}{c++}
#include<iostream>
#include<stdlib.h>
#include<vector>
#include<list>
#include<map>
//通过使用标准模板库,学习list用法
using namespace std;
int main()
{
  list<int>list1;
  list1.push_back(4);
  list1.push_back(7);
  list1.push_back(10);
  /*for(int i=0;i<list1.size();i++)
  {
    cout<<list1[i]<<endl;
  }*/    //有误,必须使用迭代器进行访问
  list<int>::iterator itor=list1.begin();
  for(;itor!=list1.end();itor++)
  {
    cout<<*itor<<endl;//遍历list数组
  }
  return 0;
}
\end{minted}
代码实例三(map用法):
\begin{minted}{c++}
#include<iostream>
#include<stdlib.h>
#include<vector>
#include<list>
#include<map>
#include<string>
//通过使用标准模板库,学习map其用法
using namespace std;
int main()
{
  /*map<int,string> m;
  pair<int,string>p1(3,"hello");
  pair<int,string>p2(6,"world");
  m.insert(p1);
  m.insert(p2);
  //cout<<m[3]<<endl;
  //cout<<m[6]<<endl;//通过key找到相应的value
  map<int,string>::iterator itor=m.begin();//迭代器
  for(;itor!=m.end();itor++)
  {
    cout<<itor->first<<endl;
    cout<<itor->second<<endl;
    cout<<endl;
  }*/
  map<string,string> m;
  pair<string,string>p1("H","hello");
  pair<string,string>p2("W","world");
  pair<string,string>p3("B","beijing");
  m.insert(p1);
  m.insert(p2);
  m.insert(p3);
  //cout<<m["H"]<<endl;
  //cout<<m["B"]<<endl;//通过key找到相应的value
  map<string,string>::iterator itor=m.begin();//迭代器
  for(;itor!=m.end();itor++)
  {
    cout<<itor->first<<endl;
    cout<<itor->second<<endl;
    cout<<endl;
  }
    return 0;
}
\end{minted}
\end{CJK}
\end{document}