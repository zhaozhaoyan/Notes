\documentclass{article}  
\usepackage{CJKutf8}
\usepackage{minted}
\begin{document} 
\hfuzz=\maxdimen
\tolerance=10000
\hbadness=10000
\begin{CJK}{UTF8}{gbsn}  
\title{第五章  继承和派生}
\author{}
\date{}
\maketitle
\part*{一、继承和派生的概念}
\section*{1.基本概念}
\subsection*{(1)继承和派生}
\subparagraph*{}
继承:在定义一个新的类B时,如果该类与某个已有的类A相似(指的B拥有A的全部特点),那么就可以把A作为一个基类,而把B作为基类的一个派生类(也称子类)。
\subparagraph*{}
派生类是通过对基类进行修改和扩充得到的。在派生类中,可以扩充新的成员变量和成员函数。
\subparagraph*{}
派生类一经定义后,可以独立使用,不依赖于基类。
\subparagraph*{}
派生类拥有基类的全部成员函数和成员变量,不论是private,protected,public。
\subparagraph*{}
1)在派生类的各个成员函数中,不能访问基类中的private成员。
\subparagraph*{}
2)学籍管理程序:学号、姓名、性别、成绩为共同属性;是否该留级,是否该奖励为共同方法(成员函数);大学生,研究生,导师等为不同的属性和方法等
\subsection*{(2)派生类的写法}
\begin{minted}{c++}
    class 派生类名:public 基类名
    {
        ......
        
    };
\end{minted}
\subsection*{(3)派生类对象的内存空间}
\subparagraph*{}
派生类对象的体积,等于基类对象的体积,再加上派生类对象自己的成员变量的体积。在派生类对象中,包含着基类对象,而且基类对象的存储位置位于派生类对象新增的成员变量之前。
\begin{minted}{c++}
    class CBase
    {
        int v1,v2;  
    };
    class CDerived:public CBase
    {
        int v3;
    }
\end{minted}
\section*{2.程序实例}
下面看一个有两个类的简单学生管理程序:
\begin{minted}{c++}
#include<iostream>
#include<string>
using namespace std;
class CStudent
{
private:
  string name;
  string id;
  char gender;
  int age;
public:
  void PrintInfo();
  void SetInfo(const string&name_,const string & id_,int age_,char gender_);
  string GetName(){return name;}
};
class CUndergraduateStudent:public CStudent
{
private:
  string department;
public:
  void QualifiedForBaoyan(){
    cout<<"qualified for baoyan"<<endl;
  }
void PrintInfo() {
  CStudent::PrintInfo();
  cout<<"Department:"<<department<<endl;
}
void SetInfo(const string&name_,const string & id_,int age_,char gender_,const string&department_)
{
   CStudent::SetInfo(name_,id_,age_,gender_);
   department=department_;
}
};
void CStudent::PrintInfo()
{
  cout<<"Name:"<<name<<endl;
  cout<<"ID:"<<id<<endl;
  cout<<"Age:"<<age<<endl;
  cout<<"Gender:"<<gender<<endl;
}
void CStudent::SetInfo(const string&name_,const string & id_,int age_,char gender_)
{
  name=name_;
  id=id_;
  age=age_;
  gender=gender_;
}
int main()
{
  CStudent s1;
  CUndergraduateStudent s2;
  s2.SetInfo("Harry Potter","118829212",19,'M',"Computer Science");
  cout<<s2.GetName()<<endl;
  s2.QualifiedForBaoyan();
  s2.PrintInfo();
  cout<<"sizeof(string)="<<sizeof(string)<<endl;
  cout<<"sizeof(CStudent)="<<sizeof(CStudent)<<endl;
  cout<<"sizeof(CUndergraduateStudent)="<<sizeof(CUndergraduateStudent)<<endl;
  return 0;
}
\end{minted}
\end{CJK}
\end{document}