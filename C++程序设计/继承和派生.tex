\documentclass{article}  
\usepackage{CJKutf8}
\usepackage{minted}
\begin{document} 
\hfuzz=\maxdimen
\tolerance=10000
\hbadness=10000
\begin{CJK}{UTF8}{gbsn}  
\title{第五章  继承和派生}
\author{}
\date{}
\maketitle
\part*{一、继承和派生的概念}
\section*{1.基本概念}
\subsection*{(1)继承和派生}
\subparagraph*{}
继承:在定义一个新的类B时,如果该类与某个已有的类A相似(指的B拥有A的全部特点),那么就可以把A作为一个基类,而把B作为基类的一个派生类(也称子类)。
\subparagraph*{}
派生类是通过对基类进行修改和扩充得到的。在派生类中,可以扩充新的成员变量和成员函数。
\subparagraph*{}
派生类一经定义后,可以独立使用,不依赖于基类。
\subparagraph*{}
派生类拥有基类的全部成员函数和成员变量,不论是private,protected,public。
\subparagraph*{}
1)在派生类的各个成员函数中,不能访问基类中的private成员。
\subparagraph*{}
2)学籍管理程序:学号、姓名、性别、成绩为共同属性;是否该留级,是否该奖励为共同方法(成员函数);大学生,研究生,导师等为不同的属性和方法等
\subsection*{(2)派生类的写法}
\begin{minted}{c++}
    class 派生类名:public 基类名
    {
        ......
        
    };
\end{minted}
\subsection*{(3)派生类对象的内存空间}
\subparagraph*{}
派生类对象的体积,等于基类对象的体积,再加上派生类对象自己的成员变量的体积。在派生类对象中,包含着基类对象,而且基类对象的存储位置位于派生类对象新增的成员变量之前。
\begin{minted}{c++}
    class CBase
    {
        int v1,v2;  
    };
    class CDerived:public CBase
    {
        int v3;
    }
\end{minted}
\section*{2.程序实例}
下面看一个有两个类的简单学生管理程序:
\begin{minted}{c++}
#include<iostream>
#include<string>
using namespace std;
class CStudent
{
private:
  string name;
  string id;
  char gender;
  int age;
public:
  void PrintInfo();
  void SetInfo(const string&name_,const string & id_,int age_,char gender_);
  string GetName(){return name;}
};
class CUndergraduateStudent:public CStudent
{
private:
  string department;
public:
  void QualifiedForBaoyan(){
    cout<<"qualified for baoyan"<<endl;
  }
void PrintInfo() {
  CStudent::PrintInfo();
  cout<<"Department:"<<department<<endl;
}
void SetInfo(const string&name_,const string & id_,int age_,char gender_,const string&department_)
{
   CStudent::SetInfo(name_,id_,age_,gender_);
   department=department_;
}
};
void CStudent::PrintInfo()
{
  cout<<"Name:"<<name<<endl;
  cout<<"ID:"<<id<<endl;
  cout<<"Age:"<<age<<endl;
  cout<<"Gender:"<<gender<<endl;
}
void CStudent::SetInfo(const string&name_,const string & id_,int age_,char gender_)
{
  name=name_;
  id=id_;
  age=age_;
  gender=gender_;
}
int main()
{
  CStudent s1;
  CUndergraduateStudent s2;
  s2.SetInfo("Harry Potter","118829212",19,'M',"Computer Science");
  cout<<s2.GetName()<<endl;
  s2.QualifiedForBaoyan();
  s2.PrintInfo();
  cout<<"sizeof(string)="<<sizeof(string)<<endl;
  cout<<"sizeof(CStudent)="<<sizeof(CStudent)<<endl;
  cout<<"sizeof(CUndergraduateStudent)="<<sizeof(CUndergraduateStudent)<<endl;
  return 0;
}
\end{minted}
\part*{二、继承关系和复合关系}
\section*{1.类之间的两种关系}
\subsection*{(1)继承关系和复合关系}
\subparagraph*{}
1.继承:“是”关系。
\subparagraph*{}
基类A,B是基类A的派生类。
\subparagraph*{}
逻辑上要求:“一个B对象也是个A对象”。
\subparagraph*{}
2.复合:“有”关系。
\subparagraph*{}
类C中“有”成员变量k,k是类D的对象,则C和D是复合关系;
\subparagraph*{}
一般逻辑上要求:“D对象是C对象的固有属性或组成部分”。
\subsection*{(2)继承关系的使用}
\subparagraph*{}
在设计两个有关系的类时,要注意,并非两个类有共同点,就可以让他们成为继承关系。让类A继承类B,必须满足“类B所代表的事物也是类A所代表的事物”,这个命题从逻辑上是成立的。
\subparagraph*{}
几何形体的绘图程序,点类和圆类,两者的关系是复合关系:
\begin{minted}{c++}
    class CCircle
    {
        CPoint center;//圆心
        double radius;//半径
    }
\end{minted}
\subsection*{(2)复合关系的使用}
\subparagraph*{}
写一个小区养狗管理程序,要有两个类,主人类和狗类:(注意不要循环定义:人中有狗,狗中有人)
\subparagraph*{}
正确的写法:是为狗类设一个主人类的指针成员变量,为主人类设一个狗类的对象数组。
\begin{minted}{c++}
    class CMaster;
    class CDog
    {
        CMaster *pm;
    };
    class CMaster{
    CDog *dogs[100];
    };
\end{minted}
\part*{三、protected访问范围说明符}
\paragraph*{}
基类的private成员:可以被下列函数访问
\subparagraph*{}
1.基类的成员函数
2.基类的友元函数
\paragraph*{}
基类的public成员:可以被下列函数访问
\subparagraph*{}
1.基类的成员函数
2.基类的友元函数
3.派生类的成员函数
4.派生类的友元函数
5.其他的函数
\paragraph*{}
基类的protected成员:可以被下列函数访问
\subparagraph*{}
1.基类的成员函数
2.基类的友元函数
3.派生类的成员函数可以访问当前对象的基类的保护成员
\subparagraph*{}
在基类中,一般都将需要隐藏的成员说明为保护成员而非私有成员。
\part*{四、派生类的构造函数}
\section*{派生类的构造函数}
\subparagraph*{}
1.派生类对象包含基类对象
\subparagraph*{}
2.执行派生类构造函数之前,先执行基类的构造函数
\subparagraph*{}
3.派生类交代基类初始化,具体形式:
\begin{minted}{c++}
    构造函数表(形参表):基类名(基类构造函数实参表)
    {
          ......
    }
\end{minted}
\subparagraph*{}
4.派生类对象消亡时,先执行派生类的析构函数,再执行基类的析构函数
\subparagraph*{}
5.在创建派生类的对象时:
\subparagraph*{}
1)需要调用基类的构造函数:初始化派生类对象中从基类继承的成员
\subparagraph*{}
2)在执行一个派生类的构造函数之前,总是先执行基类的构造函数
\subparagraph*{}
6.调用基类构造函数的两种方式:
\subparagraph*{}
1)显示方式:派生类的构造函数中—>基类的构造函数提供参数
\subparagraph*{}
2)隐式方式:派生类的构造函数中,省略基类构造函数时,派生类的构造函数,自动调用基类的默认构造函数
\subparagraph*{}
派生类的析构函数被执行时,执行完派生类的析构函数后,自动调用基类的析构函数
\section*{2.派生类的构造函数和析构函数调用顺序}
\begin{minted}{c++}
#include<iostream>
using namespace std;
class Base{
 public:
  int n;
  Base(int i):n(i)
  {
    cout<<"Base "<<n<<" constructed"<<endl;//1
  }
  ~Base()
  {
    cout<<"Base "<<n<<" destructed"<<endl;//2
  }
};
class Derived:public Base{//派生类构造函数
 public:
  Derived(int i):Base(i)//基类构造函数
  {
    cout<<"Derived constructed"<<endl;//4
  }
  ~Derived()//基类的析构函数
  {
    cout<<"Derived destructed"<<endl;//3
  }
};
int main()
{
  Derived Obj(3);
  return 0;
}
\end{minted}
\subparagraph*{}
1.创建派生类的对象时,执行派生类的构造函数之前:
\subparagraph*{}
1)调用基类的构造函数:初始化派生类对象中从基类继承的成员
\subparagraph*{}
2)调用成员对象类的构造函数:初始化派生类对象中的成员对象
\subparagraph*{}
2.执行完派生类的析构函数后:
\subparagraph*{}
1)调用成员对象类的析构函数
\subparagraph*{}
2)调用基类的函数
\subparagraph*{}
注意:析构函数的调用顺序与构造函数的调用顺序相反
\part*{五、public继承的赋值兼容规则}
\subparagraph*{}
(1)派生类的对象可以赋值给基类对象;
\subparagraph*{}
(2)派生类的对象可以用来初始化基类引用;
\subparagraph*{}
(3)派生类对象的地址可以赋值给基类指针,亦即派生类的指针可以赋值给基类的指针。
\subparagraph*{}
以上三条反过来是不成立的。例如,不能把基类对象赋值给派生类对象。
\subparagraph*{}
在公有派生的情况下,可以说,派生类对象也是基类对象,任何本该出现基类对象的地方,如果出现的是派生类的对象,也是没有问题的。
\end{CJK}
\end{document}